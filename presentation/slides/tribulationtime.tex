\section{Tempo tribulo e Análise de Risco}

\begin{frame}
  \frametitle{Tempo tribulo e Análise de Risco}
  \begin{itemize}
    \item O tempo tribulo é importante para a análise de risco porque é necessário considerar o passado para identificar ameaças e vulnerabilidades potenciais
    \item A análise de risco envolve a identificação e avaliação de possíveis ameaças e vulnerabilidades e a tomada de medidas para mitigar ou eliminar esses riscos
    \item O objetivo da análise de risco é garantir a segurança e a integridade dos sistemas e dados de uma organização e proteger contra ameaças potenciais, como ataques cibernéticos e violações de dados
  \end{itemize}
\end{frame}

\begin{frame}
  \frametitle{Análise de Risco Completa}
  \begin{itemize}
    \item Para realizar uma análise de risco completa, é importante considerar o passado, o presente e o futuro
    \item Isso significa observar incidentes de segurança anteriores e suas causas, bem como tendências e desenvolvimentos atuais no campo da segurança cibernética
    \item Ao fazer isso, as organizações podem entender melhor os tipos de ameaças que podem enfrentar e podem tomar medidas para prevenir ou mitigar essas ameaças no futuro
  \end{itemize}
\end{frame}

\begin{frame}
  \frametitle{Exemplos de Análise de Risco}
  \begin{itemize}
    \item Se uma organização sofreu uma violação de dados no passado, pode analisar a causa dessa violação e tomar medidas para evitar que incidentes semelhantes ocorram no futuro
    \item Isso pode incluir a implementação de novas medidas de segurança, como senhas mais fortes ou autenticação de dois fatores, ou a atualização de protocolos de segurança para melhor proteção contra ameaças potenciais
    \item A análise de risco também pode envolver a realização de avaliações de risco regulares e o mantenimento de atualizações com as últimas tendências e práticas recomendadas no campo da segurança cibernética
  \end{itemize}
\end{frame}
