\section{ISO 27002}
\begin{frame}
  \frametitle{Pilares da ISO 27002}
  \begin{itemize}
    \item \textbf{Tecnologia} - ferramentas e sistemas usados para proteger informações e sistemas de ameaças potenciais
    \item \textbf{Pessoas} - indivíduos dentro de uma organização responsáveis por implementar e manter a segurança da informação
    \item \textbf{Segurança física} - medidas em vigor para proteger os ativos físicos de uma organização, incluindo medidas ativas e passivas
    \item \textbf{Organização} - procedimentos e processos em vigor para garantir a gestão eficaz da segurança da informação dentro de uma organização.
  \end{itemize}
\end{frame}

\begin{frame}
  \frametitle{Tecnologia}
  \begin{itemize}
    \item Política de segurança
    \item Arquitetura de segurança
    \item Gestão de ativos
    \item Controle de acesso
    \item Criptografia
    \item Desenvolvimento e manutenção do sistema
    \item Gestão de vulnerabilidade técnica
  \end{itemize}
\end{frame}

\begin{frame}
  \frametitle{Pessoas}
  \begin{itemize}
    \item Consciencialização e treinamento de segurança
    \item Segurança do pessoal
    \item Gestão de incidentes de segurança
    \item Gestão de continuidade de negócios
  \end{itemize}
\end{frame}

\begin{frame}
  \frametitle{Segurança física}
  \begin{itemize}
    \item Política de segurança física
    \item Perímetro de segurança física
    \item Controlo de acesso físico
    \item Segurança física dos ativos
  \end{itemize}
\end{frame}

\begin{frame}
  \frametitle{Organização}
  \begin{itemize}
    \item Conformidade legal
    \item Avaliação e gestão de riscos
    \item Relacionamento com fornecedores
    \item Objetivos de segurança da informação
    \item Auditoria interna
    \item Revisão administrativa
  \end{itemize}
\end{frame}
