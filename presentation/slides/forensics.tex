\section{Intervenção Digital Forense}
\begin{frame}
  \frametitle{O que é Intervenção Digital Forense?}
  \text{A intervenção digital forense refere-se ao uso de técnicas forenses digitais para coletar, preservar e analisar evidências digitais após um incidente de segurança cibernética.}
\end{frame}

\begin{frame}
  \frametitle{Tipos de Intervenção Digital Forense}
  \text{Existem dois tipos de intervenção digital forense:}
  \begin{itemize}
    \item Intervenção forense online: envolve conduzir a investigação enquanto os sistemas afetados ainda estão online e conectados à rede. Permite coletar dados ao vivo, mas também há o risco de alteração das provas.
    \item Intervenção forense offline: envolve conduzir a investigação depois que os sistemas afetados foram colocados offline e desconectados da rede. Permite coletar um conjunto de evidências mais preciso, mas não permite interromper o ataque ou evitar mais danos.
  \end{itemize}
\end{frame}

\begin{frame}
  \frametitle{Indicadores Coletados durante a Intervenção Digital Forense}
  \text{O investigador geralmente coleta uma ampla gama de indicadores durante a intervenção digital forense para ajudar a identificar a causa do incidente e as partes responsáveis. Isso pode incluir:}
  \begin{itemize}
    \item Indicadores de Comprometimento (IOCs): amostras de malware, logs de tráfego de rede, etc.
    \item Indicadores de Ataque (IOAs): tentativas suspeitas de login, exfiltração de dados, etc.
  \end{itemize}
  \text{Os indicadores específicos coletados dependerão do incidente específico e das ferramentas e técnicas utilizadas pelo investigador.}
\end{frame}
