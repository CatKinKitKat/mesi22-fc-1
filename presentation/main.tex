% !TeX program = XeLaTeX
% !TeX encoding = UTF-8
% !TeX spellcheck = pt_PT
% !TeX root = main
\documentclass[xcolor=svgnames,t]{beamer}
\usepackage[portuguese]{babel}
\usepackage[utf8]{inputenc}
\usepackage[backend=biber,citestyle=numeric,sorting=none,style=ieee]{biblatex}
\usepackage[font=footnotesize]{caption}
\usepackage[absolute, overlay]{textpos}
\usepackage[sfdefault]{noto}
\usepackage[makeroom]{cancel}
\usepackage{booktabs, comment}
\usepackage{amsmath}
\usepackage{csquotes}
\usepackage{textpos}
\usepackage{tikz}
\usepackage{multicol}
\usepackage{eurosym}

\addbibresource{main.bib}

% school colours
\definecolor{ipbgreen}{RGB}{166,204,59}
\definecolor{ipbbrown}{RGB}{153,80,42}
\definecolor{estigred}{RGB}{150,0,24}

\usetheme{Rochester} %Modern Like God foretold
\useoutertheme{infolines}
\usecolortheme[named=estigred]{structure}
\setbeamercolor{title in head/foot}{fg=estigred, bg=}
\setbeamercolor{author in head/foot}{bg=estigred}
\setbeamertemplate{page number in head/foot}{}

% title and settings
\title[AMARO]{Trabalho Individual}
\subtitle{Fundamentos de Cibersegurança}
\institute[]{Escola Superior de Tecnologia e Gestão\\Instituto Politécnico de Beja}
\titlegraphic{\includegraphics[height=2.0cm]{estig.png}}
\author[17440]{Gonçalo Amaro -- 17440} % Doc's author/s
\institute[]{Escola Superior de Tecnologia e Gestão\\Instituto Politécnico de Beja}
\logo{\includegraphics[height=1.0cm]{estig.png}}
\date{15 de Dezembro, 2022} % Doc date

% presentation controls
\addtobeamertemplate{navigation symbols}{}{%
    \usebeamerfont{footline}%
    \usebeamercolor[fg]{footline}%
    \hspace{1em}%
    \insertframenumber/\inserttotalframenumber
}

%start
\begin{document}

%cover
{
\setbeamertemplate{headline}{} % hide useless top bar
\logo{} % hide redundant logo
\begin{frame}
  \vspace{-1cm} % damn son, looks good
  \maketitle
\end{frame}
}

%toc
\begin{frame}
  \texttt{}{A estrutura desta apresentação está organizada da seguinte forma:}
  \tableofcontents[hideallsubsections]
\end{frame}

%input slides/chapters (cut and paste to separate files later)
\chapter{Introdução}

\section{História}

\section{Princípios da encriptação quântica}

\textit{artigos: "Principios da Criptografia Quantica, Quantum Key Distribution Protocol, Ataques quanticos"}


\section{Vetores de Ataque}
\begin{frame}
  \frametitle{Vetores de Ataque}
  \begin{itemize}
    \item Os vetores de ataque são um conceito crítico na segurança cibernética porque fornecem uma maneira de os invasores obterem acesso a um sistema de computador ou rede
    \item Existem muitos tipos diferentes de vetores de ataque, incluindo ataques de phishing, malware, engenharia social, exploits, acesso físico, etc... Dependendo da granularidade da conversa, pode ser enumerado mais ou menos tipos de vetores de ataque.
    \item Entender essas ameaças é essencial para que as organizações se protejam contra elas
  \end{itemize}
\end{frame}

\begin{frame}
  \frametitle{Vetores de Ataque}
  \begin{multicols}{2}
    \begin{itemize}
      \item \textbf{Phishing (Engenharia Social via E-mail)}
      \item \textbf{Malware}
      \item \textbf{Engenharia Social (Pessoas)}
      \item \textbf{Exploração de Vulnerabilidades}
      \item \textbf{Acesso Físico}
    \end{itemize}
    \begin{itemize}
      \item \textbf{Vulnerabilidade de Software}
      \item \textbf{Vulnerabilidade de Hardware}
      \item \textbf{Vulnerabilidade de Rede}
      \item \textbf{Vulnerabilidade de comportamento humano}
    \end{itemize}
  \end{multicols}
\end{frame}

\section{Projeto ATT\&CK do MITRE}
\begin{frame}
  \frametitle{Projeto ATT\&CK do MITRE}
  \begin{itemize}
    \item O projeto ATT\&CK é uma base de conhecimento pública desenvolvida pela MITRE Corporation
    \item Fornece informações detalhadas sobre as táticas e técnicas usadas por diferentes grupos adversários em ataques cibernéticos
    \item Oferece uma linguagem comum e uma compreensão das diferentes maneiras pelas quais os invasores podem operar
    \item Ajuda as organizações a entenderem a ameaça específica que estão enfrentando e a melhorarem suas defesas
    \item Pode ser útil para a investigação de ataques cibernéticos e para a identificação de responsáveis
  \end{itemize}
\end{frame}

\begin{frame}
  \frametitle{Grupos Adversários no Projeto ATT\&CK}
  \begin{itemize}
    \item APT1 (``Comment Crew'') -- grupo de hackers patrocinado pelo estado chinês
    \item APT29 (``The Dukes' ou ``Cozy Bear'') -- grupo associado a atividades de espionagem cibernética patrocinadas pelo estado russo
  \end{itemize}
  \begin{itemize}
    \item Usam uma ampla gama de táticas, incluindo spearphishing, malware e exploração de redes
    \item Usam ferramentas e técnicas sofisticadas, incluindo ferramentas de hacking personalizadas e malware
  \end{itemize}
\end{frame}


\section{Tempo tribulo e Análise de Risco}

\begin{frame}
  \frametitle{Tempo tribulo e Análise de Risco}
  \begin{itemize}
    \item O tempo tribulo é importante para a análise de risco porque é necessário considerar o passado para identificar ameaças e vulnerabilidades potenciais
    \item A análise de risco envolve a identificação e avaliação de possíveis ameaças e vulnerabilidades e a tomada de medidas para mitigar ou eliminar esses riscos
    \item O objetivo da análise de risco é garantir a segurança e a integridade dos sistemas e dados de uma organização e proteger contra ameaças potenciais, como ataques cibernéticos e violações de dados
  \end{itemize}
\end{frame}

\begin{frame}
  \frametitle{Análise de Risco Completa}
  \begin{itemize}
    \item Para realizar uma análise de risco completa, é importante considerar o passado, o presente e o futuro
    \item Isso significa observar incidentes de segurança anteriores e suas causas, bem como tendências e desenvolvimentos atuais no campo da segurança cibernética
    \item Ao fazer isso, as organizações podem entender melhor os tipos de ameaças que podem enfrentar e podem tomar medidas para prevenir ou mitigar essas ameaças no futuro
  \end{itemize}
\end{frame}

\begin{frame}
  \frametitle{Exemplos de Análise de Risco}
  \begin{itemize}
    \item Se uma organização sofreu uma violação de dados no passado, pode analisar a causa dessa violação e tomar medidas para evitar que incidentes semelhantes ocorram no futuro
    \item Isso pode incluir a implementação de novas medidas de segurança, como senhas mais fortes ou autenticação de dois fatores, ou a atualização de protocolos de segurança para melhor proteção contra ameaças potenciais
    \item A análise de risco também pode envolver a realização de avaliações de risco regulares e o mantenimento de atualizações com as últimas tendências e práticas recomendadas no campo da segurança cibernética
  \end{itemize}
\end{frame}

\section{ISO 27002}
\begin{frame}
  \frametitle{Pilares da ISO 27002}
  \begin{itemize}
    \item \textbf{Tecnologia} - ferramentas e sistemas usados para proteger informações e sistemas de ameaças potenciais
    \item \textbf{Pessoas} - indivíduos dentro de uma organização responsáveis por implementar e manter a segurança da informação
    \item \textbf{Segurança física} - medidas em vigor para proteger os ativos físicos de uma organização, incluindo medidas ativas e passivas
    \item \textbf{Organização} - procedimentos e processos em vigor para garantir a gestão eficaz da segurança da informação dentro de uma organização.
  \end{itemize}
\end{frame}

\begin{frame}
  \frametitle{Tecnologia}
  \begin{itemize}
    \item Política de segurança
    \item Arquitetura de segurança
    \item Gestão de ativos
    \item Controle de acesso
    \item Criptografia
    \item Desenvolvimento e manutenção do sistema
    \item Gestão de vulnerabilidade técnica
  \end{itemize}
\end{frame}

\begin{frame}
  \frametitle{Pessoas}
  \begin{itemize}
    \item Consciencialização e treinamento de segurança
    \item Segurança do pessoal
    \item Gestão de incidentes de segurança
    \item Gestão de continuidade de negócios
  \end{itemize}
\end{frame}

\begin{frame}
  \frametitle{Segurança física}
  \begin{itemize}
    \item Política de segurança física
    \item Perímetro de segurança física
    \item Controlo de acesso físico
    \item Segurança física dos ativos
  \end{itemize}
\end{frame}

\begin{frame}
  \frametitle{Organização}
  \begin{itemize}
    \item Conformidade legal
    \item Avaliação e gestão de riscos
    \item Relacionamento com fornecedores
    \item Objetivos de segurança da informação
    \item Auditoria interna
    \item Revisão administrativa
  \end{itemize}
\end{frame}

\section{Intervenção Digital Forense}
\begin{frame}
  \frametitle{O que é Intervenção Digital Forense?}
  A intervenção digital forense refere-se ao uso de técnicas forenses digitais para coletar, preservar e analisar evidências digitais após um incidente de segurança cibernética.
\end{frame}

\begin{frame}
  \frametitle{Tipos de Intervenção Digital Forense}
  Existem dois tipos de intervenção digital forense:
  \begin{itemize}
    \item Intervenção forense online: envolve conduzir a investigação enquanto os sistemas afetados ainda estão online e conectados à rede. Permite coletar dados ao vivo, mas também há o risco de alteração das provas.
    \item Intervenção forense offline: envolve conduzir a investigação depois que os sistemas afetados foram colocados offline e desconectados da rede. Permite coletar um conjunto de evidências mais preciso, mas não permite interromper o ataque ou evitar mais danos.
  \end{itemize}
\end{frame}

\begin{frame}
  \frametitle{Indicadores Coletados durante a Intervenção Digital Forense}
  O investigador geralmente coleta uma ampla gama de indicadores durante a intervenção digital forense para ajudar a identificar a causa do incidente e as partes responsáveis. Isso pode incluir:
  \begin{itemize}
    \item Indicadores de Comprometimento (IOCs): amostras de malware, logs de tráfego de rede, etc.
    \item Indicadores de Ataque (IOAs): tentativas suspeitas de login, exfiltração de dados, etc.
  \end{itemize}
  Os indicadores específicos coletados dependerão do incidente específico e das ferramentas e técnicas utilizadas pelo investigador.
\end{frame}

\section{Campanha de sensibilização}
\begin{frame}
  \frametitle{Campanha de sensibilização}
  \text{A campanha pode incluir uma variedade de atividades de consciencialização, como:}
  \begin{itemize}
    \item Fornecer treinamento e educação regulares aos funcionários sobre a importância da segurança cibernética.
    \item Partilhando exemplos de incidentes de segurança cibernética da vida real e as medidas tomadas para evitá-los.
    \item Destacando as possíveis consequências de não levar a segurança cibernética a sério, como perdas financeiras, danos à reputação e responsabilidades legais.
    \item Incentivar os funcionários a relatar quaisquer atividades suspeitas ou possíveis ameaças.
    \item Além disso, eu me certificaria de monitorizar e avaliar regularmente a eficácia da campanha e fazer os ajustes necessários.
  \end{itemize}
\end{frame}


\clearpage
\nocite{*}
\section{Referências}
\printbibliography[heading=bibintoc, title={Referências}]

\end{document}
