\chapter{Conclusão}

Em conclusão, a pesquisa e análise neste artigo mostraram que existem inúmeros vetores de ataque no âmbito da segurança cibernética. Esses vetores de ataque incluem técnicas como phishing, malware e engenharia social, que podem ser usadas por invasores para obter acesso a um sistema ou rede. O Projeto ATT\&CK da Mitre é uma estrutura abrangente que visa identificar e classificar esses vetores de ataque e fornecer orientação sobre como se defender deles. O projeto é relevante no contexto da cibersegurança e do combate ao cibercrime porque ajuda as organizações a entender as táticas, técnicas e procedimentos utilizados por diferentes grupos de invasores.

Um grupo de invasores de particular interesse é o grupo APT \textit{(Advanced Persistent Threat)}. Esse grupo é conhecido por sua capacidade de conduzir ataques furtivos e contínuos a um alvo, geralmente com o objetivo de extrair dados confidenciais. A escolha deste grupo justificou-se porque os ataques APT são muitas vezes difíceis de detetar e podem causar danos significativos a uma organização. Os exemplos de enitidades APT usados foram os grupos APT1 e APT29, que são conhecidos por seus ataques contra organizações governamentais e empresas privadas, com suporte e promoção de governos estrangeiros. A análise dos vetores de ataque e das técnicas de defesa do grupo APT1 e APT29 ajudou a identificar as principais táticas, técnicas e procedimentos utilizados por esses grupos e a fornecer orientação sobre como se defender deles.

Em termos de tempo de tribulação e análise de risco, é importante considerar o passado, o presente e o futuro ao avaliar a probabilidade e o impacto de possíveis incidentes de segurança. Ao entender as tendências e padrões de ataques anteriores, as organizações podem planear e se preparar melhor para possíveis ameaças futuras. Isso pode ajudar a enriquecer o processo de análise de risco e melhorar a postura geral de segurança de uma organização.

A versão atual da \textbf{ISO/IEC 27002} (2022) identifica quatro pilares da segurança de informação: Tecnologia, pessoas, segurança física e Organização. Cada pilar tem um conjunto de subprincípios que fornecem orientação sobre como implementar medidas de segurança eficazes nessa área.

Em um incidente de segurança cibernética, indicadores de comprometimento (IOCs) e indicadores de ataque (IOAs) podem ser fontes valiosas de informações para intervenções forenses digitais. Esses indicadores podem ajudar os investigadores a determinar o escopo e a gravidade de um ataque e identificar possíveis culpados. O tipo de indicadores coletados durante uma intervenção dependerá se a intervenção é realizada online ou offline. As intervenções online podem coletar indicadores relacionados ao tráfego de rede e logs do sistema, enquanto as intervenções offline podem coletar indicadores de dispositivos físicos e midia.

Para desenhar e implementar uma ação de sensibilização eficaz numa entidade privada, é importante ter em consideração os tipos de incidentes de cibersegurança mais comuns na Europa. Isso pode incluir incidentes como ataques de phishing e ataques de ransomware, que têm aumentado nos últimos anos. A ação deve ter como objetivo educar os funcionários sobre essas ameaças e fornecer a eles o conhecimento e as ferramentas necessárias para proteger a si mesmos e à organização de tais ataques.

No geral, este documento fornece uma visão abrangente dos principais tópicos relacionados à segurança cibernética e à luta contra o crime cibernético. A pesquisa e a análise apresentadas neste documento podem ajudar as organizações a entender melhor e se defender contra os vários vetores de ataque que representam uma ameaça a seus sistemas e redes. Ao adotar uma abordagem pro-ativa e holística à segurança, as organizações podem melhorar sua resiliência a ameaças cibernéticas e proteger seus ativos e informações valiosos.

\newpage
