\section[Pergunta 2]{Atualmente, na actual da \textbf{ISO 27002}. quantos são e quais são, os pilares da Segurança da Informação?}

O padrão \textbf{\textbf{ISO/IEC 27002}} é um padrão internacional amplamente reconhecido que fornece diretrizes e princípios gerais para iniciar, implementar, manter e melhorar o gestão de segurança da informação em uma organização. Faz parte da família de normas \textbf{ISO/IEC 27000}, que é uma série de normas que fornecem diretrizes, melhores práticas e princípios gerais para o gestão da segurança da informação.

Foi publicado pela primeira vez em 1995 como \textbf{ISO/IEC 17799}, e foi revisado e renomeado como \textbf{ISO/IEC 27002} em 2013 (posteriormente em 2022). O padrão é baseado no Código de Prática para Gestão de Segurança da Informação, desenvolvido pela Organização Internacional de Normalização (ISO) e Comissão Eletrotécnica Internacional (IEC).

Este fornece uma estrutura para as organizações usarem ao desenvolver e implementar políticas e controles de segurança da informação. O padrão é baseado no princípio de que a segurança da informação deve ser integrada ao gestão geral de uma organização, em vez de ser tratada como uma preocupação separada.

\textbf{Está dividido em duas secções principais.} A primeira secção fornece uma visão geral do gestão de segurança da informação e os princípios que devem ser seguidos ao implementar controles de segurança da informação. A segunda secção fornece uma descrição detalhada dos dez pilares da estrutura \textbf{ISO/IEC 27002}, que são projetados para fornecer uma abordagem abrangente para o gestão da segurança da informação.

É apenas um dos vários padrões que fazem parte da família \textbf{ISO/IEC 27000}. A família de padrões \textbf{ISO/IEC 27000} fornece diretrizes, melhores práticas e princípios gerais para o gestão de segurança da informação. Alguns dos outros padrões da família \textbf{ISO/IEC 27000} incluem:

\begin{itemize}
  \item \textbf{ISO/IEC 27001:} Esta norma especifica os requisitos para estabelecer, implementar, manter e melhorar continuamente um sistema de gestão de segurança da informação (SGSI).
  \item \textbf{ISO/IEC 27003:} Esta norma fornece orientação sobre a implementação de um SGSI baseado na \textbf{ISO/IEC 27001}.
  \item \textbf{ISO/IEC 27004:} Esta norma fornece diretrizes para a medição e avaliação da eficácia de um SGSI.
  \item \textbf{ISO/IEC 27005:} Esta norma fornece orientação sobre o gestão de riscos de segurança da informação com base na \textbf{ISO/IEC 27001}.
  \item \textbf{ISO/IEC 27006:} Esta norma especifica os requisitos para organizações que fornecem certificação de ISMSs com base na \textbf{ISO/IEC 27001}.
  \item \textbf{ISO/IEC 27007:} Esta norma fornece diretrizes para auditoria de sistemas de gestão de segurança da informação.
  \item \textbf{ISO/IEC 27008:} Esta norma fornece diretrizes para a implementação de controles de segurança da informação em serviços de terceiros de processamento de informações.
  \item \textbf{ISO/IEC 27010:} Este padrão fornece orientação sobre a coordenação da segurança da informação em organizações que possuem vários ISMSs.
\end{itemize}

Esses padrões são projetados para ajudar as organizações a implementar sistemas eficazes de gestão de segurança da informação e para fornecer uma estrutura e terminologia comuns para o gestão de segurança da informação. Juntos, os padrões da família \textbf{ISO/IEC 27000} fornecem uma abordagem abrangente para gerir a segurança da informação e são amplamente utilizados por organizações em todo o mundo.

Este padrão, \textbf{ISO/IEC 27002}, contrasta o padrão \textbf{ISO/IEC 27001}, que especifica os requisitos para estabelecer, implementar, manter e melhorar continuamente um sistema de gestão de segurança da informação (SGSI). O padrão \textbf{ISO/IEC 27002} fornece orientação sobre como implementar controles de segurança da informação em uma organização e é uma estrutura amplamente utilizada para estabelecer, implementar, manter e melhorar continuamente a gestão de segurança da informação dentro de uma organização. \textbf{A norma fornece um conjunto de diretrizes e princípios gerais para iniciar, implementar, manter e melhorar a gestão de segurança da informação em uma organização.}

A norma é dividida em três grupos principais: \textbf{áreas, princípios e pilares.}

As \textbf{áreas} do padrão \textbf{ISO 27002} abrangem as principais áreas de preocupação com a segurança da informação, incluindo \textbf{segurança humana, segurança física e segurança lógica} (também conhecida como \textbf{segurança cibernética}).

\textbf{A segurança humana} abrange medidas para proteger os indivíduos dentro de uma organização de possíveis ameaças à sua segurança, como assédio, intimidação ou outras formas de abuso.

\textbf{A segurança física} envolve medidas para proteger os ativos físicos de uma organização, como seus prédios, equipamentos e data centers, contra possíveis ameaças. A segurança física pode ser dividida em medidas ativas e passivas. Medidas ativas são aquelas que protegem ativamente contra ameaças, como guardas de segurança e câmaras de vigilância. Medidas passivas são aquelas que protegem contra ameaças sem fazê-lo ativamente, como fechaduras e cercas.

\textbf{A segurança lógica}, também conhecida como \textbf{segurança cibernética}, abrange medidas para proteger as informações e os sistemas de uma organização contra possíveis ameaças. Isso inclui medidas como \textit{firewalls}, criptografia e políticas de senha segura.

Os \textbf{princípios} do padrão \textbf{ISO 27002} são as diretrizes subjacentes para estabelecer e manter a segurança da informação dentro de uma organização. Existem quatro princípios principais: \textbf{confidencialidade, disponibilidade, integridade e não-repúdio.}

\textbf{Confidencialidade} refere-se à proteção de informações confidenciais contra divulgação não autorizada. Isso significa que apenas indivíduos autorizados devem ter acesso a informações confidenciais e que as informações devem ser mantidas em segredo, a menos que haja um motivo legítimo para compartilhá-las.

\textbf{Disponibilidade} refere-se à capacidade de indivíduos autorizados de acessar informações e sistemas quando necessário. Isso significa que as informações e os sistemas devem estar disponíveis e funcionando o tempo todo, a menos que haja um motivo legítimo para sua indisponibilidade.

\textbf{Integridade} refere-se à precisão e integridade das informações. Isso significa que as informações devem ser precisas, completas e atualizadas em todos os momentos, e que quaisquer alterações nas informações devem ser feitas de maneira controlada e autorizada.

\textbf{O não-repúdio} refere-se à capacidade de provar que uma determinada ação foi realizada por um indivíduo específico. Isso é importante nos casos em que a autenticidade de uma ação, como enviar um e-mail ou fazer uma transação financeira, precisa ser verificada. A perícia digital desempenha um papel no não repúdio, ajudando a identificar o indivíduo responsável por uma determinada ação.

Os \textbf{pilares} da norma \textbf{ISO 27002} são os quatro principais componentes de um sistema eficaz de gestão de segurança da informação. Esses pilares são \textbf{tecnologia, pessoas, segurança física e organização.}

\textbf{Tecnologia} refere-se às ferramentas e sistemas usados para proteger as informações e os sistemas de uma organização contra ameaças potenciais. Isso inclui hardware e software, como \textit{firewalls}, sistemas de deteção de intrusão e algoritmos de criptografia.

\textbf{As pessoas} referem-se aos indivíduos dentro de uma organização que são responsáveis pela implementação e manutenção da segurança da informação. Isso inclui funcionários, contratados e provedores de serviços de terceiros que têm acesso a informações ou sistemas confidenciais.

\textbf{A segurança física} refere-se às medidas em vigor para proteger os ativos físicos de uma organização, conforme discutido acima. Isso inclui medidas ativas e passivas, como guardas de segurança e fechaduras.

\textbf{Organização} refere-se aos procedimentos e processos em vigor para garantir que a segurança da informação seja efetivamente gerida dentro de uma organização. Isso inclui políticas, padrões e diretrizes para segurança da informação, bem como as funções, responsabilidades e treinamento dos indivíduos envolvidos na segurança da informação.

\newpage
