\section[Pergunta 2]{Apresente o Projeto ATT\&CK do Mitre; qual a sua relevância no âmbito da Cibersegurança e do Combate ao Cibercrime, relativamente aos Grupos de atacantes de eu forma este Projeto pretende contribuir para a identificação dos atacantes: escolha ainda um Grupo enunciado e apresente-o, justifique a escolha desse Grupo.}

\textbf{O projeto \textit{Adversarial Tactics, Techniques, and Common Knowledge} (ATT\&CK) da MITRE Corporation é uma estrutura abrangente para entender as táticas, técnicas e procedimentos (TTPs) usados por adversários em ataques cibernéticos.} Desenvolvido pela organização sem fins lucrativos MITRE, que opera centros de pesquisa e desenvolvimento para o governo dos EUA, o projeto ATT\&CK é uma base de conhecimento disponível publicamente que fornece informações detalhadas sobre as táticas e técnicas usadas por diferentes grupos adversários.

\textbf{O projeto ATT\&CK é relevante para a segurança cibernética porque fornece uma linguagem comum e uma compreensão das diferentes maneiras pelas quais os invasores podem operar, o que pode ajudar as organizações a se defenderem desses ataques.} Ao fornecer informações detalhadas sobre as táticas e técnicas utilizadas por diferentes grupos adversários, o projeto ATT\&CK permite que as organizações entendam melhor a ameaça específica que estão enfrentando e melhorem suas defesas contra esses ataques.

Uma das principais características do projeto ATT\&CK é seu foco em diferentes grupos de invasores, conhecidos como ``grupos adversários''. Esses grupos são definidos com base nas táticas e técnicas que eles usam em seus ataques e fornecem uma maneira para as organizações entenderem a ameaça específica que estão enfrentando. Por exemplo, o APT1, também conhecido como \textit{``Comment Crew''}, é um grupo de hackers patrocinado pelo estado Chinês que está ativo desde pelo menos 2006. Esse grupo é conhecido por usar uma ampla gama de táticas, incluindo \textit{spearphishing}, malware e rede exploração, para realizar espionagem cibernética e roubo de propriedade intelectual.

Outro grupo que está incluído no projeto ATT\&CK é o APT29, também conhecido como \textit{``The Dukes''} ou \textit{``Cozy Bear''}. Acredita-se que esse grupo esteja associado a atividades de espionagem cibernética patrocinadas pelo estado Russo e esteja ativo desde pelo menos 2008. O APT29 é conhecido por usar ferramentas e técnicas sofisticadas, incluindo ferramentas de hacking personalizadas e malware, para comprometer seus alvos.

A escolha sobre esses dois grupos para o exemplo dá-se ao facto que são grupos suportados e patrocinados por Estados, o que aumenta o meu fascínio pessoal sobre as suas ações e torna a sua ameaça mais perigosa e mais difícil de combater, visto que os Estados têm recursos financeiros e humanos para desenvolver e implementar estratégias de ataque ultra sofisticadas e complexas como também garantir a sua proteção e anonimato. Torna-se difícil para as organizações privadas e governamentais combaterem esses grupos.

\textbf{Além de fornecer informações sobre grupos adversários, o projeto ATT\&CK também inclui descrições detalhadas das táticas e técnicas que esses grupos utilizam em seus ataques}. Por exemplo, o projeto inclui informações sobre táticas como \textit{spearphishing} e malware, bem como técnicas como exploração de rede e quebra de senha. Ao fornecer essas informações detalhadas, o projeto ATT\&CK permite que as organizações entendam melhor as táticas e técnicas específicas que diferentes grupos adversários usam e desenvolvam defesas eficazes contra esses ataques.

No geral, o projeto MITRE ATT\&CK é um recurso valioso para organizações que buscam melhorar suas defesas contra ataques cibernéticos. Ao fornecer informações detalhadas sobre táticas, técnicas e procedimentos usados por diferentes grupos adversários, o projeto ATT\&CK permite que as organizações entendam melhor as ameaças específicas que enfrentam e desenvolvam defesas eficazes contra esses ataques.

\newpage
