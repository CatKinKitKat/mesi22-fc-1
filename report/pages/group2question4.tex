\section[Pergunta 4]{Num incidente de Cibersegurança, em que medida os `IOCs' e os `IOAs' podem ser salvaguardados por uma intervenção digital forense? Havendo dois tipos de intervenção habituais (em linha e fora de linha) quais são os indicadores tipicamente recolhidos por um e por outro tipo de intervenção?}

Em um incidente de segurança cibernética, indicadores de comprometimento (IOCs) e indicadores de ataque (IOAs) podem ser protegidos até certo ponto por meio de intervenção digital forense. A intervenção digital forense refere-se ao uso de técnicas forenses digitais para coletar, preservar e analisar evidências digitais após um incidente de segurança cibernética.

Existem dois tipos de intervenção digital forense: online e offline. A intervenção forense online envolve conduzir a investigação forense enquanto os sistemas afetados ainda estão online e conectados à rede. Isso permite que o investigador colete dados ao vivo e potencialmente interrompa o ataque em andamento. No entanto, também acarreta o risco de adulteração ou alteração das provas.

A intervenção forense offline envolve a condução da investigação forense depois que os sistemas afetados foram colocados offline e desconectados da rede. Isso permite que o investigador colete um conjunto de evidências mais completo e preciso, pois os sistemas não estão mais sendo atacados ativamente. No entanto, isso também significa que a investigação pode não ser capaz de interromper o ataque ou evitar mais danos.

Tanto na intervenção forense online quanto offline, o investigador geralmente coleta uma ampla gama de indicadores para ajudar a identificar a causa do incidente e as partes responsáveis. Isso pode incluir IOCs, como amostras de malware ou logs de tráfego de rede, bem como IOAs, como tentativas suspeitas de login ou exfiltração de dados.

Os indicadores específicos coletados durante uma intervenção digital forense dependerão do incidente específico e das ferramentas e técnicas utilizadas pelo investigador. Em geral, a intervenção forense online pode se concentrar na coleta de dados ao vivo, como tráfego de rede e logs do sistema, enquanto a intervenção forense offline pode se concentrar na coleta de artefactos dos sistemas afetados, como arquivos e chaves de registo.

No geral, a intervenção digital forense pode ajudar a proteger IOCs e IOAs, coletando e preservando evidências do incidente, que podem ser usadas para identificar a causa e as partes responsáveis. No entanto, a eficácia da intervenção dependerá do incidente específico, das habilidades e experiência do investigador e da disponibilidade de ferramentas e técnicas apropriadas.

\subsection[Extensão da pergunta 4]{Como estão relacionados os princípios da segurança da informação com a intervenção digital forense?}

A intervenção digital forense está relacionada a dois dos quatro princípios da segurança da informação: integridade e não repúdio.

O princípio da integridade refere-se à precisão e integridade das informações. No contexto da intervenção digital forense, isto significa que as provas recolhidas durante a investigação devem ser precisas e completas, e que quaisquer alterações ou alterações às provas devem ser efetuadas de forma controlada e autorizada.

O princípio do não repúdio refere-se à capacidade de provar que uma determinada ação foi realizada por um indivíduo específico. No contexto da intervenção digital forense, isto significa que as provas recolhidas durante a investigação devem ser suficientes para identificar os responsáveis e provar o seu envolvimento no incidente.

Assim, seguindo os princípios de integridade e não repúdio, a intervenção forense digital pode ajudar a garantir que as evidências coletadas durante a investigação sejam precisas e completas, e que possam ser usadas para identificar os responsáveis e responsabilizá-los por seus atos.

\newpage
