\chapter{Perguntas do Grupo 1}

\section[Pergunta 1]{Pesquise, enumere e argumente sobre “Vetores de Ataque” no âmbito da Cibersegurança.}

Os vetores de ataque são um conceito crítico na segurança cibernética porque fornecem uma maneira de os invasores obterem acesso a um sistema de computador ou rede. Existem muitos tipos diferentes de vetores de ataque que os invasores podem usar, e entender essas ameaças é essencial para que as organizações se protejam contra elas.

\textbf{Um tipo comum de vetor de ataque é um ataque de \textit{phishing}.} Em um ataque de \textit{phishing}, o invasor envia um e-mail ou outra mensagem que parece ser de uma fonte legítima, como um banco ou outra empresa conhecida. A mensagem geralmente inclui um link ou anexo no qual o usuário é incentivado a clicar. Se o usuário cair no golpe e clicar no link ou abrir o anexo, ele pode ser levado a um site falso que parece real, mas na verdade é controlado pelo invasor. O usuário pode ser solicitado a inserir informações confidenciais (como credenciais de login) no site falso, que o invasor pode usar para obter acesso às contas do usuário.

\textbf{Outro tipo de vetor de ataque é o malware.} Malware é um software projetado especificamente para danificar ou interromper um sistema de computador. Existem muitos tipos diferentes de malware, incluindo vírus, worms e ransomware. O malware pode ser distribuído por meio de uma variedade de vetores de ataque, incluindo anexos de e-mail, sites maliciosos e unidades USB infetadas. Depois que o malware é instalado no sistema, ele pode executar uma ampla variedade de ações, como excluir arquivos, roubar informações confidenciais ou assumir o controle do sistema.

\textbf{Ataques de engenharia social são outro tipo comum de vetor de ataque.} Em um ataque de engenharia social, o invasor usa manipulação psicológica para induzir o usuário a divulgar informações confidenciais ou realizar uma determinada ação. Isso pode incluir táticas como fingir ser um representante de atendimento ao cliente de uma empresa conhecida ou ligar para o usuário e alegar ser do banco. O invasor pode usar várias táticas para tentar ganhar a confiança do usuário, como fornecer informações convincentes ou criar um senso de urgência. Uma vez que o usuário foi induzido a fornecer ao invasor informações confidenciais, o invasor pode usar essas informações para obter acesso às contas ou sistemas do usuário.

\textbf{\textit{Exploits} são outro tipo de vetor de ataque que os invasores podem usar.} Um \textit{exploit} é um pedaço de software ou código que tira proveito de uma vulnerabilidade em um sistema (como software sem patch) para obter acesso não autorizado. Por exemplo, se um sistema tiver uma vulnerabilidade conhecida que permite que um invasor obtenha acesso sem uma senha, um invasor pode usar um \textit{exploit} para tirar proveito dessa vulnerabilidade e obter acesso ao sistema.

\textbf{Por fim, o acesso físico é outro tipo de vetor de ataque que os invasores podem usar.} Em um ataque de acesso físico, o invasor obtém acesso físico a um computador ou rede para contornar as medidas de segurança e obter acesso ao sistema. Por exemplo, um invasor pode roubar um laptop do escritório de um funcionário ou obter acesso a uma sala de servidores fingindo ser um funcionário da manutenção. Uma vez que o invasor tenha acesso físico ao sistema, ele pode usá-lo para obter acesso à rede e roubar informações confidenciais.

\subsection[Extensão à Pergunta 1]{Pesquise, enumere e argumente sobre ``Vulnerabilidades'' no âmbito da Cibersegurança.}

Vulnerabilidades de segurança cibernética referem-se a pontos fracos ou falhas em sistemas de computador, redes ou software que podem ser explorados por invasores para obter acesso não autorizado ou realizar atividades maliciosas. Existem vários tipos de vulnerabilidades que podem existir em um sistema de segurança cibernética, incluindo:

\begin{itemize}
  \item \textbf{Vulnerabilidades de software: }são falhas ou pontos fracos em programas de software que podem ser explorados por invasores para obter acesso a um sistema ou realizar atividades maliciosas. Exemplos de vulnerabilidades de software incluem estouros de buffer, falhas de injeção de SQL e brechas de segurança não corrigidas.
  \item \textbf{Vulnerabilidades de hardware: }são falhas ou pontos fracos em dispositivos de hardware, como roteadores, servidores e outros dispositivos de rede, que podem ser explorados por invasores para obter acesso a um sistema ou rede. Exemplos de vulnerabilidades de hardware incluem senhas fracas ou padrão, pontos de acesso físicos não seguros e protocolos de segurança inadequados.
  \item \textbf{Vulnerabilidades de rede: }são falhas ou fraquezas em uma rede que podem ser exploradas por invasores para obter acesso a um sistema ou rede. Exemplos de vulnerabilidades de rede incluem redes sem fio não seguras, configurações de rede mal configuradas e protocolos inseguros.
  \item \textbf{Vulnerabilidades humanas: }são vulnerabilidades que surgem das ações ou comportamentos dos indivíduos dentro de uma organização. Exemplos de vulnerabilidades humanas incluem senhas fracas, manuseio descuidado de informações confidenciais e cair em golpes de phishing.
\end{itemize}

É importante identificar e lidar com essas vulnerabilidades para proteger contra ataques cibernéticos e manter a segurança dos sistemas de computador, redes e software. Ao implementar fortes medidas de segurança, corrigir e atualizar regularmente o software e fornecer treinamento e educação aos usuários, as organizações podem reduzir sua exposição a vulnerabilidades de segurança cibernética e se proteger contra possíveis ataques.

\newpage
