\section[Pergunta 3]{Cada Pilar da Segurança da Informação tem um conjunto de subprincípios; enuncie os mesmos.}

A norma \textbf{ISO 27002} divide a segurança da informação em quatro pilares principais: \textbf{tecnologia, pessoas, segurança física e organização.} Cada pilar possui um conjunto de subprincípios que fornecem diretrizes mais específicas para implementar e manter a segurança da informação dentro de uma organização.

O pilar de \textbf{tecnologia} inclui os seguintes subprincípios:

\begin{itemize}
  \item \textbf{Política de segurança: }uma organização deve ter uma política de segurança por escrito que defina sua abordagem à segurança da informação e que seja revisada e atualizada regularmente.
  \item \textbf{Arquitetura de segurança: }uma organização deve ter uma arquitetura de segurança que defina os controles técnicos e as medidas implementadas para proteger suas informações e sistemas.
  \item \textbf{Gestão de ativos: }Convém que uma organização mantenha um registro de seus ativos, incluindo informações e sistemas, e deve revisar e atualizar regularmente o registo.
  \item \textbf{Controle de acesso: }uma organização deve ter controles de acesso para garantir que apenas pessoas autorizadas tenham acesso a informações e sistemas confidenciais.
  \item \textbf{Criptografia: }uma organização deve usar criptografia para proteger informações confidenciais, como encriptar dados em repouso e em trânsito.
  \item \textbf{Desenvolvimento e manutenção do sistema: }Uma organização deve ter processos em vigor para desenvolver e manter seus sistemas de informação, incluindo práticas de codificação segura e patches e atualizações regulares.
  \item \textbf{Gestão de vulnerabilidade técnica: }Uma organização deve ter processos em vigor para identificar, avaliar e mitigar vulnerabilidades técnicas em seus sistemas de informação.
\end{itemize}

O pilar de pessoas inclui os seguintes subprincípios:

\begin{itemize}
  \item \textbf{Consciencialização e treinamento de segurança: }Convém que uma organização forneça consciencialização e treinamento de segurança a seus funcionários, contratados e outros indivíduos com acesso a informações e sistemas confidenciais.
  \item \textbf{Segurança do pessoal: }Convém que uma organização tenha processos em vigor para verificar a identidade e os antecedentes de indivíduos com acesso a informações e sistemas confidenciais.
  \item \textbf{Gestão de incidentes de segurança: }Uma organização deve ter processos para detetar, responder e recuperar de incidentes de segurança.
  \item \textbf{Gestão de continuidade de negócios: }uma organização deve ter planos para manter as operações de negócios no caso de um desastre ou outra interrupção.
\end{itemize}

O pilar de segurança física inclui os seguintes subprincípios:

\begin{itemize}
  \item \textbf{Política de segurança física: }uma organização deve ter uma política de segurança física por escrito que defina sua abordagem para proteger seus ativos físicos e que seja revisada e atualizada regularmente.
  \item \textbf{Perímetro de segurança física: }Uma organização deve ter um perímetro de segurança física que defina os limites de sua área protegida e que seja protegida por medidas apropriadas, como cercas, portões e guardas de segurança.
  \item \textbf{Controle de acesso físico: }Convém que uma organização tenha processos para controlar o acesso a suas instalações e ativos físicos, como por meio do uso de cartões de acesso e crachás de segurança.
  \item \textbf{Segurança física dos ativos: }uma organização deve ter medidas para proteger seus ativos físicos de ameaças potenciais, como fechaduras, alarmes e câmaras de vigilância.
\end{itemize}

O pilar da organização inclui os seguintes subprincípios:

\begin{itemize}
  \item \textbf{Conformidade legal: }uma organização deve cumprir todas as leis e regulamentos relevantes relacionados à segurança da informação.
  \item \textbf{Avaliação e gestão de riscos: }uma organização deve avaliar regularmente seus riscos relacionados à segurança da informação e deve ter processos em vigor para gerir e mitigar esses riscos.
  \item \textbf{Relacionamento com fornecedores: }Convém que uma organização tenha processos para gerir seus relacionamentos com fornecedores e outros provedores de serviços de terceiros, incluindo a avaliação de suas práticas de segurança da informação.
  \item \textbf{Objetivos de segurança da informação: }Uma organização deve ter objetivos específicos, mensuráveis, atingíveis, relevantes e com prazos (SMART) relacionados à segurança da informação e deve monitorizar e relatar seu progresso para atingir esses objetivos.
  \item \textbf{Auditoria interna: }Uma organização deve ter uma função de auditoria interna que revise e avalie regularmente suas práticas de segurança da informação e faça recomendações para melhorias.
  \item \textbf{Revisão administrativa: }Uma organização deve ter um processo de revisão administrativa para garantir que seu sistema de gestão de segurança da informação seja eficaz e alinhado com as metas e objetivos gerais da organização.
\end{itemize}

\newpage
